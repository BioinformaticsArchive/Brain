\documentclass{article}
\usepackage[a4paper,vmargin={10mm,10mm},hmargin={10mm,10mm}]{geometry}

\begin{document}

\section*{Supplementary Material}
  
\begin{table}[h]
\caption{Example of some common OWL 2EL constructs written using the
Manchester syntax alonside an example of implementation using Brain.}
\centering
{\begin{tabular}{c|c|c|c}
\hline\hline
\textbf{Name} & \textbf{Description Logic} & \textbf{OWL (Manchester Syntax)} & \textbf{Brain implementation}\\[1ex] \hline
\multicolumn{4}{c}{\textbf{Concepts}}\\[2ex] \hline
atomic concept & A & Class: A & brain.addClass("A");\\[1ex]
intersection & C\sqcap D & C and D & brain.equivalentClasses("A","C and D");\\[1ex]
top concept & \top & owl:Thing & brain.getOWLClass("Thing");\\[1ex]
bottom concept & \bot & owl:Nothing & brain.getUnsatisfiableClasses();\\[1ex]
union & C\sqcup D & C or D & Not supported (Not in EL profile)\\[1ex]
complement & \neg C & not C & Not supported (Not in EL profile)\\[1ex]
existential restriction & \exists R.C & P some C & Not supported (Not in EL profile)\\[1ex]
universal restriction & \forall R.C & P only C & Not supported (Not in EL profile)\\[1ex] \hline
\multicolumn{4}{c}{\textbf{Roles}}\\[2ex] \hline
atomic role & R & ObjectProperty: P & brain.addObjectProperty("P");\\[1ex] \hline
\multicolumn{4}{c}{\textbf{Individuals}}\\[2ex] \hline
individual name & a & Individual: a & Not supported yet\\[1ex] \hline
\multicolumn{4}{c}{\textbf{Axioms}}\\[2ex] \hline
\multicolumn{4}{c}{TBox (terminological axioms)}\\[1.5ex] \hline
concept inclusion & C \sqsubseteq D & C SubClassOf: D & brain.subClassOf("C", "D");\\[1ex]
concept equivalence & C \equiv D & C EquivalentTo: D & brain.equivalentClasses("C", "D");\\[1ex]
concept disjointness & C\sqcap D \sqsubseteq \bot & C DisjointWith: D & brain.disjointClasses("C", "D");\\[1ex] \hline
\multicolumn{4}{c}{RBox (relational axioms)}\\[1.5ex] \hline
role inclusion & R \sqsubseteq S & R SubPropertyOf: S & brain.subPropertyOf("R", "S");\\[1ex]
role equivalence & R \equiv S & R EquivalentTo: S & brain.equivalentProperties("R", "S");\\[1ex]
complex role inclusion & R1\circ R2 \sqsubseteq S & S SubPropertyChain: R1 o R2 & brain.chain("R1 o R2", "S");\\[1ex]
role transitivity & R \circ R \sqsubseteq R & Characteristics: Transitive & brain.transitive("R");\\[1ex] \hline
\multicolumn{4}{c}{ABox (assertional axioms)}\\[1.5ex] \hline
concept assertion & C(a) & a Types: C & Not supported yet\\[1ex]
role assertion & R(a, b) & a Facts: R b & Not supported yet\\[1ex]
individual equality & a = b & a SameAs: b & Not supported yet\\[1ex]
individual inequality & a \neq b & a DifferentFrom: b & Not supported yet\\[1ex]
\hline\hline
\end{tabular}}{}
\end{table}

\begin{table}[h]
\caption{The same query involving implicit knowledge retrieval is formulated using SQL and OWL over the Gene Ontology (GO).
The original data comes from the 
website of the GO: http://www.geneontology.org/GO.downloads.database.shtml
}
\centering
{\begin{tabular}{c|c|c}
\hline\hline

 & \textbf{SQL} & \textbf{OWL (via Brain)} \\[1ex] \hline
 
Source & go\_daily-termdb-tables/ & go\_daily-termdb.owl \\[1ex] \hline

Access & http://www.berkeleybop.org/goose/ & brain.learn("go\_daily-termdb.owl"); \\[1ex] \hline

\begin{tabular}[c]{@{}c@{}}
Query: \\ Explicit and implicit\\ regulators of 'blood coagulation'
\end{tabular} &
\begin{tabular}[c]{@{}c@{}}
SELECT DISTINCT * FROM term\\
 INNER JOIN graph\_path AS g \\
 ON (term.id=g.term1\_id \\
 AND g.relationship\_type\_id=21) \\
 INNER JOIN term AS r \\
 ON (r.id=g.term2\_id)\\
WHERE term.name='blood coagulation' \\
AND distance \textless \textgreater \vspace{0cm} 0 ;\\ 
\end{tabular} &
\begin{tabular}[c]{@{}c@{}}
brain.getSubClasses( \\ ``RO\_0002211 some GO\_0007596",\\ false); 
\end{tabular} \\[1ex] \hline

Method of retrieval & 
\begin{tabular}[c]{@{}c@{}}
Iteration over \\ additional tables storing all \\ the possible links (closure graph)
\end{tabular} &
\begin{tabular}[c]{@{}c@{}}
Automated reasoning \\ over the knowledge base \\ using an abstract OWL expression
\end{tabular} \\[1ex] \hline

\hline\hline
\end{tabular}}{} 
\end{table}
\end{document}
