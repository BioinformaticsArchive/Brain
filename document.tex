\documentclass{bioinfo}
\copyrightyear{2005}
\pubyear{2005}

\begin{document}
\firstpage{1}

\title[short Title]{Brain: Biomedical Knowledge Manipulation}
\author[Sample \textit{et~al}]{Samuel Croset\,$^{1,\footnote{to whom correspondence should be addressed}}$, 
Robert Hoehndorf\,$^{2}$, John Overington\,$^{1}$ and Dietrich Rebholz-Schuhmann\,$^{1}$}
\address{$^{1}$EMBL European Bioinformatics Institute, Wellcome Trust Genome Campus, Cambridge, CB10 1SD UK\\
$^{2}$Department of Genetics, University of Cambridge, Downing Street, Cambridge, CB2 3EH, UK}

\history{Received on XXXXX; revised on XXXXX; accepted on XXXXX}

\editor{Associate Editor: XXXXXXX}

\maketitle

\begin{abstract}

\section{Summary:}
Brain is a software library facilitating the creation and manipulation of ontologies and knowledge-bases represented in the
Web Ontology Language (OWL).

\section{Availability and Implementation:}
The Java source code and the library are freely available at https://github.com/loopasam/Brain
and on the Maven Central repository (GroupId: uk.ac.ebi.brain).
The documentation is available at https://github.com/loopasam/Brain/wiki.

\section{Contact:}
\href{croset@ebi.ac.uk}{croset@ebi.ac.uk}

\section{Supplementary information:}
Links to additional figures/data available on a web site, or 
reference to online-only Supplementary data available at the journal's web site.

\end{abstract}
\section{Motivation}
Relational databases currently hold most of the available structured biomedical information. The content of these repositories is often
extracted from scientific literature by manual curation with the help of text-mining tools. The transformation from raw text into
structured data is a key step, since the curated information can then be classified, managed and queried more easily. Databases facilitate
the re-use of previous work in a computer-friendly manner and support large-scale mining of
biomedical knowledge. In order to leverage further 
the existing information, the current trend is towards coordinated data integration and interoperability, with large projects such as 
ELIXIR \citep{Crosswell2012} leading the way.
The underlying idea assumes that more complex biomedical challenges, such as finding new treatments for diseases, could be better 
addressed by combining and integrating the content of independent specialized repositories. 
Knowledge-bases, a concept coming from computer science (see \citealp{Krotzsch2012} for an introduction), 
could be a solution to improve the interoperability and the value of the data yet no framework is available at the time of writing for 
the biomedical domain. Brain - the software library presented in this manuscript addresses this matter and provides a simplified 
interface dedicated to handle and query biomedical knowledge-bases.

\section{Knowledge-bases}
Traditional relational databases are often an obstacle in realizing large-scale data integration, 
mostly because of their lack of support for 
interoperability: The schema structuring the data is usually repository-specific which limits the combination of the 
native content with external data in an efficient, flexible and meaningful fashion. 
In order to address this issue, 
a series of community standards forming the \emph{semantic web} have been developed. These standards provide the means to better model
the domain knowledge and to facilitate data exchange and interoperability at a global scale.
One of them, the Resources Description 
Framework (RDF) enables the exposition of the underlying structure as part of the data themselves \citep{FrankManola}. This
representation relies on triples as building block, composed as \emph{subject - relation - object} which serves to
describe the data as well as their types. More complicated data structures, such as sub-class or
transitive relationships can be further expressed via another standard, the Web Ontology Language (OWL). OWL derives from description logic and
has been designed to capture the knowledge of a domain of interest in the form of a structured vocabulary \citep{W3COWLWorkingGroup}. 
This feature makes it particularly 
interesting from the perspective of the life sciences, since a number of ontologies and classification schemes
have been developed from the origin of 
the discipline. OWL is often expressed in combination with RDF in order to reveal the underlying schema of the data, but 
can also be used as such, as an implementation of description logic for computers. 
Knowledge-bases and ontologies can therefore be built without being necessarily expressed as RDF triples while still preserving all the 
advantages in regards to data integration and interoperability.
Brain aims at facilitating the construction and manipulation of 
such knowledge-bases or ontologies, rather than being oriented towards the consumption of RDF data. 
We will present first the biomedical motivation for the particular subset of OWL supported by Brain - so called OWL 2EL, 
then will be discussed the main features implemented by the library.

\section{Scalability and complexity}
Knowledge representation in the biomedical domain differs from other disciplines, because of the diversity and abundance of
information that can potentially be interconnected. Brain appreciates this aspect and focuses on a particular profile of OWL, called EL
which consists of a subset of the constructs available in the original language \citep{Motik2009}. This profile is
designed to be \emph{tractable}, meaning that the constructs available have a polynomial complexity, easier to compute than the full 
version of OWL. 
These constructs are called \emph{axioms} and are the fundamental notion behind OWL knowledge-bases.
Axioms assert the facts and relations present in the knowledge-base and are computable by a program named \emph{reasoner}. 
Based on the logical structure of the axioms, the reasoner is capable of deriving new facts from the asserted ones as well as retrieving
some implicit information, enabling powerful query mechanisms surpassing the Structured Query Language (SQL). 
Brain primarily supports the OWL 2EL profile for its computational properties, and is suitable for real-life biomedical applications, 
where millions of axioms could be potentially extracted from complex repositories such as ChEMBL \citep{Gaulton2012}.
Moreover, the EL profile is expressive enough to cover a good portion of biomedical knowledge: Most of the ontologies such as 
the Gene Ontology (GO - \citealp{Ashburner2000}) or 
the Chemical Entities of Biological Interest (ChEBI - \citealp{DeMatos2010}) ontology are already included in this 
profile, and any relational model can be easily 
converted and represented using OWL 2EL, opening doors for large-scale meaningful data integration.
Brain builds on the top of Elk, a fast reasoner dedicated to EL ontologies \citep{YevgenyKazakov2011}. Elk shows good 
performances at handling 
large datasets and offers the possibility to run some reasoning tasks in parallel, therefore
clusters or multicore architecture can scale the speed of reasoning as more data are added to the knowledge-base.
Brain wraps and simplifies the interaction with Elk while still leaving the possibility to fine tune 
the configuration for advanced users.

\section{Programmatic features}
One purpose of Brain is to ease the manipulation of knowledge-bases as well as increasing the rate at 
which they can be developed and validated.
In practice, the implementation of OWL ontologies and knowledge-bases can be done in either a programmatic way 
via the OWL-API \citep{MatthewHorridge2011} or with the
help of a visual tool such as Protege \citep{StanfordCenterforBiomedicalInformaticsResearch}. 
The graphical interface of Protege allows the user to focus 
on the generation of axioms rather than on details of the underlying programmatic implementation:
OWL expressions can be entered 
with the user-friendly Manchester syntax \citep{Horridge2006} and the name of the classes are displayed in a convenient way. 
More complex applications requiring a deeper control over the ontology
can take advantage the OWL-API but requires proficiency in Java and can be daunting for casual users.
The API deals with all the aspects around ontology generation for semantic web purposes nonetheless the 
level of granularity can be cumbersome for some of the biomedical applications. Brain aims at filling the gap between the OWL-API
and graphical interfaces: It is implemented as a facade, providing a series of convenience methods for common
use-cases encountered in the biomedical domain and leveraging the access to the OWL-API. 
The full list of currently supported constructs is available in the supplementary material.
Brain relies on the intuitive and explicit Manchester syntax to formulate OWL class expressions, just like in the Protege editor.
The interaction with the OWL-API centers around strings handling rather than Java objects, making Brain suitable to parse and answer requests in 
the context of a web service or an OWL end-point for instance. Using strings as input speeds as well the production 
and flexibility of the code written to move to an OWL representation from a relational or flat file database for instance.
The library supports the loading and referencing of external ontologies in order to integrate and reason
over data from different sources. An important feature of Brain is the query mechanism, which is similar to 
that implemented in Protege. Powerful questions can be formulated over the knowledge-base using the Manchester syntax, abstracting
away complex interaction with the Java object provided by the OWL-API. An example of question answering over the GO using Brain
is compared against a traditional SQL query in the supplementary material.

\section{Conclusion}
Brain is an open-source Java library designed to build and query biomedical knowledge-bases or OWL ontologies.
The library is centered on the EL profile, designed to be suitable and scalable for life-science knowledge representation. 
The convenience methods provided by Brain should simplify the development of biomedical knowledge-bases and allow developers
to increase their productivity while dealing with data integration challenges.

\section*{Acknowledgement}
\paragraph{Funding\textcolon}
This work was supported by …
 
%  \bibliographystyle{natbib}
%  \bibliographystyle{achemnat}
%  \bibliographystyle{plainnat}
%  \bibliographystyle{abbrv}
%  \bibliographystyle{bioinformatics}
% 
%  \bibliographystyle{plain}
% 
%  \bibliography{document}
 
\begin{thebibliography}{}

\bibitem[Ashburner {\em et~al.}(2000)Ashburner, Ball, Blake, Botstein, Butler,
  Cherry, Davis, Dolinski, Dwight, Eppig, Harris, Hill, Issel-Tarver,
  Kasarskis, Lewis, Matese, Richardson, Ringwald, Rubin, and
  Sherlock]{Ashburner2000}
Ashburner, M., Ball, C.~A., Blake, J.~A., Botstein, D., Butler, H., Cherry,
  J.~M., Davis, A.~P., Dolinski, K., Dwight, S.~S., Eppig, J.~T., Harris,
  M.~A., Hill, D.~P., Issel-Tarver, L., Kasarskis, A., Lewis, S., Matese,
  J.~C., Richardson, J.~E., Ringwald, M., Rubin, G.~M., and Sherlock, G.
  (2000).
\newblock {Gene Ontology: tool for the unification of biology}.
\newblock {\em Nature Genetics\/}, {\bf 25}(1), 25--29.

\bibitem[Crosswell and Thornton(2012)Crosswell and Thornton]{Crosswell2012}
Crosswell, L.~C. and Thornton, J.~M. (2012).
\newblock {ELIXIR: a distributed infrastructure for European biological data.}
\newblock {\em Trends Biotechnol.}, {\bf 30}, 241----242.

\bibitem[{De Matos} {\em et~al.}(2010){De Matos}, Alc\'{a}ntara, Dekker, Ennis,
  Hastings, Haug, Spiteri, Turner, and Steinbeck]{DeMatos2010}
{De Matos}, P., Alc\'{a}ntara, R., Dekker, A., Ennis, M., Hastings, J., Haug,
  K., Spiteri, I., Turner, S., and Steinbeck, C. (2010).
\newblock {Chemical Entities of Biological Interest: an update}.
\newblock {\em Nucleic Acids Research\/}, {\bf 38}(Database issue), D249--D254.

\bibitem[Gaulton {\em et~al.}(2012)Gaulton, Bellis, Bento, Chambers, Davies,
  Hersey, Light, McGlinchey, Michalovich, Al-Lazikani, and
  Overington]{Gaulton2012}
Gaulton, A., Bellis, L.~J., Bento, A.~P., Chambers, J., Davies, M., Hersey, A.,
  Light, Y., McGlinchey, S., Michalovich, D., Al-Lazikani, B., and Overington,
  J.~P. (2012).
\newblock {ChEMBL: a large-scale bioactivity database for drug discovery}.
\newblock {\em Nucleic Acids Research\/}, {\bf 40}(Database issue), D1100--7.

\bibitem[{Horridge} and {Bechhofer}(2011){Horridge} and
  {Bechhofer}]{MatthewHorridge2011}
{Horridge} and {Bechhofer} (2011).
\newblock {The OWL API: A Java API for OWL Ontologies.}
\newblock {\em Semantic Web Journal\/}, pages 11--21.

\bibitem[Horridge {\em et~al.}(2006)Horridge, Drummond, Goodwin, Rector,
  Stevens, and Wang]{Horridge2006}
Horridge, M., Drummond, N., Goodwin, J., Rector, A., Stevens, R., and Wang,
  H.~H. (2006).
\newblock {The Manchester OWL Syntax}.
\newblock {\em Syntax\/}, {\bf 216}, 10--11.

\bibitem[{Kazakov} {\em et~al.}(2011){Kazakov}, {Markus Kr\"{o}tzsch}, and
  {Franti\v{s}ek Siman\v{c}\'{\i}k}]{YevgenyKazakov2011}
{Kazakov}, {Markus Kr\"{o}tzsch}, and {Franti\v{s}ek Siman\v{c}\'{\i}k} (2011).
\newblock {Concurrent Classification of EL Ontologies}.
\newblock {\em Proceedings of the 10th International Semantic Web Conference
  (ISWC'11)\/}, {\bf 7032}.

\bibitem[Kr\"{o}tzsch {\em et~al.}(2012)Kr\"{o}tzsch, Simanˇ, and
  Horrocks]{Krotzsch2012}
Kr\"{o}tzsch, M., Simanˇ, F., and Horrocks, I. (2012).
\newblock {A Description Logic Primer}.
\newblock {\em Language\/}, {\bf abs/1201.4}(January), 1--16.

\bibitem[{Manola} and {Miller}(2004){Manola} and {Miller}]{FrankManola}
{Manola} and {Miller} (2004).
\newblock {RDF Primer}.

\bibitem[Motik {\em et~al.}(2009)Motik, Grau, Horrocks, Wu, Fokoue, and
  Lutz]{Motik2009}
Motik, B., Grau, B.~C., Horrocks, I., Wu, Z., Fokoue, A., and Lutz, C. (2009).
\newblock {OWL 2 Web Ontology Language Profiles}.
\newblock {\em Language\/}, {\bf 2009}(October), 1--53.

\bibitem[{Stanford Center for Biomedical Informatics Research}(2012){Stanford
  Center for Biomedical Informatics
  Research}]{StanfordCenterforBiomedicalInformaticsResearch}
{Stanford Center for Biomedical Informatics Research} (2012).
\newblock {Prot\'{e}g\'{e} Project}.

\bibitem[{W3C OWL Working Group}(2009){W3C OWL Working
  Group}]{W3COWLWorkingGroup}
{W3C OWL Working Group} (2009).
\newblock {OWL 2 Web Ontology Language Document Overview}.

\end{thebibliography}

\end{document}
