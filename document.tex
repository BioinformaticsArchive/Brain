\documentclass{bioinfo}
\copyrightyear{2005}
\pubyear{2005}

\begin{document}
\firstpage{1}

\title[short Title]{Brain: Biomedical Knowledge Manipulation}
\author[Sample \textit{et~al}]{Samuel Croset\,$^{1,\footnote{to whom correspondence should be addressed}}$, 
Robert Hoehndorf\,$^{2}$, John Overington\,$^{1}$ and Dietrich Rebholz-Schuhmann\,$^{1}$}
\address{$^{1}$European Bioinformatics Institute, Wellcome Trust Genome Campus, Cambridge, CB10 1SD UK\\
$^{2}$Department of Genetics, University of Cambridge, Downing Street, Cambridge, CB2 3EH, UK}

\history{Received on XXXXX; revised on XXXXX; accepted on XXXXX}

\editor{Associate Editor: XXXXXXX}

\maketitle

\begin{abstract}

\section{Summary:}
Brain is a library facilitating the creation and manipulation of ontologies and knowledge bases represented in the
Web Ontology Language (OWL).

\section{Availability and Implementation:}
The Java source code and the library are freely available at https://github.com/loopasam/Brain
and on the Maven Central repository (GroupId: uk.ac.ebi.brain).
The documentation is available at https://github.com/loopasam/Brain/wiki.

\section{Contact:}
\href{croset@ebi.ac.uk}{croset@ebi.ac.uk}

\section{Supplementary information:}
Links to additional figures/data available on a web site, or 
reference to online-only Supplementary data available at the journal's web site.

\end{abstract}
\section{Motivation}
Relational databases hold most of the available structured biomedical information. The content of these repositories is often
extracted from scientific literature by manual curation with the help of text-mining tools. The transformation from raw text into
structured data is most important, as the curated information can then be classified, managed and queried more easily. Databases facilitate
the re-use of previous work in a computer-friendly manner and support the biomedical knowledge to scale-up. In order to leverage further 
more the existing information, the current trend is at data integration and interoperability, with large projects such as 
ELIXIR \citep{elixir} leading the way.
The underlying idea assumes that increasingly complex biomedical challenges such as finding new treatments for diseases could be better 
addressed by combining  and integrating the content of independent repositories. 
Knowledge bases \citep{DL primer}, a concept coming from computer science, 
could be a solution to improve the interoperability and the value of the data yet no framework is available at the time of writting for 
the biomedical domain. Brain, the library presented in this manuscript addresses this matter and provides a simplified 
interface dedicated to handle and query biomedical knowledge bases.

\section{Knowledge bases}
Traditional relational databases are an obstacle to realize a large-scale data integration, mostly because of their lack of support for 
interoperability: The schema structuring the data is indeed very repository-specific which limits the combination of the 
native content with external data in an efficient and meaningful fashion. In order to address this issue, 
a series a standard forming the semantic web have been developped. One of them, the Resources Description 
Framework (RDF) enables the exposition of the underlying structure as part of the data themselves \citep{rdf}. This
representation relies on triples as building block, composed as \emph{subject - relation - object} which serves to
describe the data as well as their types. More complicated data structures, such as sub-class or
transitive relationships can be further expressed via another standard, the Web Ontology Language (OWL). OWL derives from description logic and
has been designed to capture the knowledge of a domain of interest in the form of a structured vocabulary \citep{owl}. 
This feature makes it particularly 
interesting from the point of view of life science, as a lot of ontologies and classification have been devellopped since the origin of 
the discipline. OWL is often expressed in combination with RDF in order to reveal the underlying schema of the data, but 
it can also be used as such, as a computer implementation of description logic. 
Knowledge bases and ontologies can therefore be built without being necessarly expressed as RDF triples while still preserving all the 
advantages in regards to data integration and interoperability.
Brain aims at facilitating the construction and manipulation of 
such knowledge bases (referred also as ontologies in this document), rather than being oriented towards the consumption of RDF data. 
We will present first the biomedical motivation for the particular subset of OWL supported by Brain so called OWL 2EL. 
Then will be discussed the main features implemented by the library in regards to this profile.

\section{Scalability and complexity}
Knowledge representation in the biomedical domain differs from other discipline, because of the type and abundance of
information that can possibly be interconnected. Brain appreciates this aspect and focuses on a particular profile of OWL, called EL
which consists of a subset of the constructs available in the original language \citep{el profiles}. This profile is
designed to be \emph{tractable}, meaning that the constructs available have a polynomial complexity, easier to compute than the full 
version of OWL. 
These constructs are called \emph{axioms} and are the fundamental block behind OWL knowledge bases.
Axioms assert the facts and relations present in the knowledge base and are understood by a computer program named \emph{reasoner}. 
Based on the logical structure of the axioms, the reasoner is indeed capable of deriving new facts from the asserted ones as well as retrieving
some implicit information, enabling powerful query mechanisms surpassing the Structured Query Language (SQL) expressivity. 
Brain is supporting primarily the OWL 2EL profile for its computational properties, as it is suitable for real-life biomedical applications, 
where millions of axioms could be potentially extracted from a repository such as Chembl.
Moreover, the profile is expressive enough for a good portion of the biomedical knowledge: Most of the ontologies such as the Gene Ontology or 
Chebi are already included in this profile and any relational model can be easely converted and represented using OWL 2EL, opening doors
for large scale meaningful data integration.
Brain builds on the top of Elk, a fast reasonner dedicated to EL ontologies. Elk shows very good performances at classifying 
large datasets and is implemented in a multi-threaded friendly way. It indeed offers the possibility to run some reasoning tasks in parallel.
Clusters or multicore architecture can therefore scale the speed of reasoning as more data are added to the knowledge base, common
situtation for biomedical repositories. Brain wraps and simplifies the interaction with Elk while still leaving the possibility to fine tune 
the configuration for advanced users.

\section{Programmatic features}
A purpose of Brain is to ease the manipulation of knowledge bases as well as increasing the production at which they can be developed.
In practice, the implementation of OWL ontologies and knowledge bases can be done either in a programmatic way via the OWL-API or with the
help of a visual tool such as Protege (itself built from the OWL-API). The graphical interface of Protege allows the user to focus 
plainly on the generation of axioms rather than on the underlying programmatic implementation:
OWL expressions can indeed be entered 
with the user-friendly Manchester syntax and the name of the classes are displayed in a convenient way for instance. 
More complex applications requiring a deeper control over the ontology
can take advantage the OWL-API but it requires a fairly good understanding of Java and can be daunting for the new comers.
The API deals indeed in great details with all the aspects around ontology generation for semantic web purposes, nonetheless the 
level of detail can be cumbersome for some of the biomedical applications. Brain aims at filling the gap between the OWL-API
and graphical interfaces: It is designed as a facade, providing a series of convenience methods for the common
use-cases encountered in the biomedical domain and leveraging the access to the OWL-API. 
Table \ref{Tab:constructs} highlights some typical OWL constructs 
with an example of implementation using Brain. The full list of supported constructs is available as supplementary material.
\begin{table}[!h]
\processtable{Example of some common OWL 2EL constructs written using the
Manchester syntax alonside an example of implementation using Brain.\label{Tab:constructs}}
{\begin{tabular}{llll}\toprule
OWL & Brain implementation\\\midrule
Class: A & brain.addClass("A");\\
ObjectProperty: P & brain.addObjectProperty("P");\\
C and D & brain.getEquivalentClasses("C and D");\\
owl:Thing & brain.getOWLClass("Thing");\\
P some C & brain.subClassOf("X", "P some C");\\
C SubClassOf: D & brain.subClassOf("C", "D");\\
S SubPropertyChain: R1 o R2 & brain.chain("R1 o R2", "S");\\
R Characteristics: Transitive & brain.transitive("R");\\\botrule
\end{tabular}}{}
\end{table}
Brain relies on the intuitive and explicit Manchester syntax to formulate OWL class expressions, just like in the Protege editor.
The interaction with the OWL-API resolves around strings rather than Java objects, making Brain suitable to parse and answer requests in 
the context of a web service or an OWL end-point for instance. Using strings of characters as input speeds as well the production 
and flexibility of the code written to move to an OWL representation from a relational or flat file database for instance.
The library supports the loading and referencing of external ontologies in order to integrate and reason
over data coming from different sources. An important feature of Brain is the query mechanism, which is very similar to 
the one implemented in Protege. Powerful questions can be formulated over the knowledge base using the Manchester syntax, abstracting
away a complex interaction with the Java object provided by the OWL-API.

\section{Conclusion}
Brain is a open-source Java library dedicated to users willing to build biomedical knowledge bases or OWL ontologies. 
The library is centered on the EL profile, designed to be suitable and scalable for life science knowledge representation. 
The convenience methods provided by Brain should simplify the development of biomedical knowledge bases and allow developers
to increase their productivity while dealing with data integration challenges.

\section*{Acknowledgement}
blabalba
\paragraph{Funding\textcolon}
This work was supported by …
 
% \bibliographystyle{natbib}
% \bibliographystyle{achemnat}
% \bibliographystyle{plainnat}
% \bibliographystyle{abbrv}
% \bibliographystyle{bioinformatics}
% 
% \bibliographystyle{plain}
% 
% \bibliography{document}


\end{document}
