\documentclass{bioinfo}
\copyrightyear{2005}
\pubyear{2005}

\begin{document}
\firstpage{1}

\title[short Title]{Brain: Biomedical Knowledge Manipulation}
\author[Sample \textit{et~al}]{Samuel \,$^{1,\footnote{to whom correspondence should be addressed}}$, JPO\,$^{2}$ and DRS\,$^{2}$}
\address{$^{1}$Department of XXXXXXX, Address XXXX etc.\\
$^{2}$Department of XXXXXXXX, Address XXXX etc.}

\history{Received on XXXXX; revised on XXXXX; accepted on XXXXX}

\editor{Associate Editor: XXXXXXX}

\maketitle

\begin{abstract}

\section{Summary:}
This section should summarize the purpose/novel features of the program in one or two sentences.

\section{Availability and Implementation:}
This section should state software availability if the paper focuses mainly on 
software development or on the implementation of an algorithm. Examples are:
Freely available on the web at http://www.example.org. Website implemented in Perl, MySQL and 
Apache, with all major browsers supported'; or 'Source code and binaries freely available for download at URL, 
implemented in C++ and supported on linux and MS Windows'. The complete address (URL) should be given. If the manuscript 
describes new software tools or the implementation of novel algorithms the software must be freely available to non-commercial 
users. Authors must also ensure that the software is available for a full TWO YEARS following publication. The editors of 
Bioinformatics encourage authors to make their source code available and, if possible, to provide access through an open source 
license see www.opensource.org for examples.


\section{Contact:}
\href{croset@ebi.ac.uk}{croset@ebi.ac.uk}

\section{Supplementary information:}
Links to additional figures/data available on a web site, or 
reference to online-only Supplementary data available at the journal's web site.

\end{abstract}

\section{Introduction}

Text Text Text Text Text Text  Text Text Text Text Text Text Text Text Text  Text Text Text Text Text Text. Figure~\ref{fig:01} shows that the above method  Text Text Text Text  Text Text Text Text Text Text  Text Text.  \citep{Bag01} wants to know about �� text follows.

\begin{equation}
\sum x+ y =Z\label{eq:01}
\end{equation}

\section{Approach}

Equation~(\ref{eq:01}) Text Text Text Text Text Text  Text Text Text Text Text Text Text Text Text  Text Text Text Text Text Text. Figure \ref{fig:02} shows that the above method  Text Text Text Text  Text Text Text Text Text Text  Text Text.  \citealp{Boffelli03} might want to know about  text text text text ��


\begin{methods}
\section{Methods}

Text Text Text Text Text Text. Figure \ref{fig:02} shows that the above method
Text Text Text Text  Text Text Text Text Text Text  Text Text.  \citealp{Boffelli03} might want

\begin{itemize}
\item for bulleted list, use itemize
\item for bulleted list, use itemize
\end{itemize}

\begin{table}[!t]
\processtable{This is table caption\label{Tab:01}}
{\begin{tabular}{llll}\toprule
head1 & head2 & head3 & head4\\\midrule
row1 & row1 & row1 & row1\\
row2 & row2 & row2 & row2\\
row3 & row3 & row3 & row3\\
row4 & row4 & row4 & row4\\\botrule
\end{tabular}}{This is a footnote}
\end{table}

\end{methods}

\begin{figure}[!tpb]%figure1
%\centerline{\includegraphics{fig01.eps}}
\caption{Caption, caption.}\label{fig:01}
\end{figure}

\begin{figure}[!tpb]%figure2
%\centerline{\includegraphics{fig02.eps}}
\caption{Caption, caption.}\label{fig:02}
\end{figure}

\section{Discussion}

discussion goes there


Table~\ref{Tab:01} shows that Text Text Text Text 

%%%%%%%%%%%%%%%%%%%%%%%%%%%%%%%%%%%%%%%%%%%%%%%%%%%%%%%%%%%%%%%%%%%%%%%%%%%%%%%%%%%%%
%
%     please remove the " % " symbol from \centerline{\includegraphics{fig01.eps}}
%     as it may ignore the figures.
%
%%%%%%%%%%%%%%%%%%%%%%%%%%%%%%%%%%%%%%%%%%%%%%%%%%%%%%%%%%%%%%%%%%%%%%%%%%%%%%%%%%%%%%

\section{Conclusion}

(Table~\ref{Tab:01}) Text Text Text Text Text Text

\begin{enumerate}
\item this is item, use enumerate
\item this is item, use enumerate
\item this is item, use enumerate
\end{enumerate}

\section*{Acknowledgement}
These should be included at the end of the text and not in footnotes. Please ensure you acknowledge all 
sources of funding, see funding section below.
Details of all funding sources for the work in question should be given 
in a separate section entitled 'Funding'. This should appear before the 'Acknowledgements' section.

\paragraph{Funding\textcolon}
The following rules should be followed:
The sentence should begin: This work was supported by …
 
The full official funding agency name should be given, i.e. National Institutes of Health, not NIH 
(full RIN-approved list of UK funding agencies) Grant numbers should be given in brackets as follows: grant number xxxx 
Multiple grant numbers should be separated by a comma as follows: grant numbers xxxx, yyyy 
Agencies should be separated by a semi-colon (plus ‘and’ before the last funding agency) 
Where individuals need to be specified for certain sources of funding the following text should be added after the
relevant agency or grant number to author initials.
Oxford Journals will deposit all NIH-funded articles in PubMed Central. See Depositing articles in repositories – 
information for authors for details. Authors must ensure that manuscripts are clearly indicated as NIH-funded using the guidelines above.

\bibliographystyle{natbib}
\bibliographystyle{achemnat}
\bibliographystyle{plainnat}
\bibliographystyle{abbrv}
\bibliographystyle{bioinformatics}

\bibliographystyle{plain}

\bibliography{document}

\end{document}
