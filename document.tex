\documentclass{llncs}
%
\usepackage{makeidx}  % allows for indexgeneration
%
\begin{document}
%
\frontmatter          % for the preliminaries
%
\pagestyle{headings}  % switches on printing of running heads

\mainmatter              % start of the contributions
%
\title{Brain, a library for the OWL2 EL profile}
%
\titlerunning{Hamiltonian Mechanics}  % abbreviated title (for running head)
%                                     also used for the TOC unless
%                                     \toctitle is used
%
\author{Samuel Croset\inst{1} \and John Overington\inst{1} \and Dietrich Rebholz-Schuhmann\inst{1}}

\institute{EMBL-EBI, Wellcome Trust Genome Campus, Hinxton, Cambridge CB10 1SD UK\\
\email{croset@ebi.ac.uk}
}
\maketitle              % typeset the title of the contribution

\begin{abstract}
Brain is a Java library facilitating the interaction with OWL2 EL ontologies. 
The library aims at bridging the gap between graphical user interfaces (GUI) such as Protege and the OWL-API: It provides 
a series of convenience methods to create and query knowledge bases using the Manchester syntax. 
The library is useful to develop web applications and particularly suited for the biomedical domain. 
Brain relies on ELK for reasoning tasks. The open source project is available at https://github.com/loopasam/Brain.

\keywords{OWL2 EL, library, Java, Manchester syntax}
\end{abstract}

\section{Motivation}

OWL2 --> profiles
profiles --> limited set of axioms, suited for a particular task
OWL2 EL --> suited for biomedical ontologies
lots of classes, very few individuals

Traditionaly interation with OWL2 is via Protege or OWL-API
no inbetweem aside OWLtools. Dedicated to fast implementation
- nothing for the web, yet it's semantic web




dedicated to the web
multi threaded environement
intetraction with strings
easy to query
presentation OWL2 EL profile.

\section{}

% ---- Bibliography ----
%
\begin{thebibliography}{5}
%
\bibitem {clar:eke}
Clarke, F., Ekeland, I.:
Nonlinear oscillations and
boundary-value problems for Hamiltonian systems.
Arch. Rat. Mech. Anal. 78, 315--333 (1982)

\bibitem {clar:eke:2}
Clarke, F., Ekeland, I.:
Solutions p\'{e}riodiques, du
p\'{e}riode donn\'{e}e, des \'{e}quations hamiltoniennes.
Note CRAS Paris 287, 1013--1015 (1978)

\bibitem {mich:tar}
Michalek, R., Tarantello, G.:
Subharmonic solutions with prescribed minimal
period for nonautonomous Hamiltonian systems.
J. Diff. Eq. 72, 28--55 (1988)

\bibitem {tar}
Tarantello, G.:
Subharmonic solutions for Hamiltonian
systems via a $\bbbz_{p}$ pseudoindex theory.
Annali di Matematica Pura (to appear)

\bibitem {rab}
Rabinowitz, P.:
On subharmonic solutions of a Hamiltonian system.
Comm. Pure Appl. Math. 33, 609--633 (1980)

\end{thebibliography}
\end{document}
