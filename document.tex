\documentclass{llncs}
%
\usepackage{makeidx}  % allows for indexgeneration
%
\begin{document}
%
\frontmatter          % for the preliminaries
%
\pagestyle{headings}  % switches on printing of running heads

\mainmatter              % start of the contributions
%
\title{Brain, a library for the OWL2 EL profile}
%
\titlerunning{Hamiltonian Mechanics}  % abbreviated title (for running head)
%                                     also used for the TOC unless
%                                     \toctitle is used
%
\author{Samuel Croset\inst{1} \and John Overington\inst{1} \and Dietrich Rebholz-Schuhmann\inst{1}}

\institute{EMBL-EBI, Wellcome Trust Genome Campus, Hinxton, Cambridge CB10 1SD UK\\
\email{croset@ebi.ac.uk}
}
\maketitle              % typeset the title of the contribution

\begin{abstract}
Brain is a Java library facilitating the interaction with OWL2 EL ontologies. 
The library aims at bridging the gap between graphical user interfaces (GUI) such as Protege and the OWL-API: It provides 
a series of convenience methods to create and query knowledge bases using the Manchester syntax. 
The library is useful to develop web applications and particularly suited for the biomedical domain. 
Brain relies on ELK for reasoning tasks. The open source project is available at https://github.com/loopasam/Brain.

\keywords{OWL2 EL, library, Java, Manchester syntax}
\end{abstract}

\section{Introduction}
\subsection{OWL2 EL}
The second version of the Web Ontology Language (OWL2) introduces a series of profiles: OWL2 EL, RL and QL.
These profiles are subsets of the full OWL2 specification and have been designed to match the requirements of particular case scenarios.
Profiles are defined by the type of axioms and constructs they support: For instance the OWL2 EL profile, which has been 
created for the biomedical domain, does not allow 
disjunctions or cardinality restrictions. This limited expressivity however enables the implementation of fast reasoning algorithms which can
handle large data as it is required by the application domain. In this document we will present the core features of a 
Java library, Brain, dedicated to support the OWL2 EL families.

\subsection{Motivation}
The biomedical domain is particularly interesting for the OWL community because of the richness and variety of the ontologies it contains.
The majority of them such as the Gene Ontology (GO) or SNOMED CT are very large but they are fortunately following 
the EL profile, which enables the use of recent and fast reasoners such as ELK.
Because of these improvements in the reasoning speed over big biomedical knowledge bases, 
it becomes nowadays possible to build web applications or to perform biological analysis relying heavely of OWL queries performed against
an ontology of interest. These taks are traditionnaly done either via a graphical user interface (GUI) such as Protege or programmed
using the OWL-API and the Java language. GUIs are wonderful to develop toy examples, but they are not suited to develop very large 
ontologies as the ones faced in life sciences, potentially containing millions of axioms. The OWL-API is solid solution to build
applications but it could be daunting for new comers or users with little experience in Java. OWLTools is an intermediary solution for the
biomedical domain, but the interaction with this library is mostly done by command lines with impairs the developement of larger projects.
In order to address these issues, we have developped Brain, a facade on the top of the OWL-API. The library facilitates the
manipulation of OWL2 EL ontologies, specially in a web server setting. 
Brain is already used in production as a back-end engine for web applications such as the Virtual Fly Brain or the Functional 
Therapeutic Chemical Classification System.

\section{Features}



\subsection{Availability}

maven, jar with dependencies
code source on github

presentation of the core features of the lib

Unique ontology

Basically, a instance of a Brain object hold a reference to only one 
ontology (also called knowledge-base). You can of course import some external 
ontologies, refer to external terms, but at the end it will resolve to only one ontology.

Unique names

The names of OWL entities handled by a Brain object have to be unique. 
This is motivated by the fact that Brain hides as much as possible the cumbersome
 interaction with prefixes, IRI and URIs. Everything resolves to names.

Typeless

The interaction with the ontology is done via strings (type-less). 
Expressions and queries are formulated in Manchester syntax. It results 
in less and more explicit code.

Error-handling driven

Because the interaction with Brain is built around strings rather than Java objects, 
special care has to be put on exception handling, in order to preserve the consistency. Brain 
throws a lot of possible exceptions, depending the type of operation you want to carry. 
All exceptions are subclasses of BrainException.

Queries

integration

\section{Example of implementation}

nucleus axiom longer 

VBF use case + FTC

\section{Conclusion}

future direction --> SVG graphs in dev, add individuals, etc..


% ---- Bibliography ----
%
\begin{thebibliography}{5}
%
\bibitem {clar:eke}
Clarke, F., Ekeland, I.:
Nonlinear oscillations and
boundary-value problems for Hamiltonian systems.
Arch. Rat. Mech. Anal. 78, 315--333 (1982)

\bibitem {clar:eke:2}
Clarke, F., Ekeland, I.:
Solutions p\'{e}riodiques, du
p\'{e}riode donn\'{e}e, des \'{e}quations hamiltoniennes.
Note CRAS Paris 287, 1013--1015 (1978)

\bibitem {mich:tar}
Michalek, R., Tarantello, G.:
Subharmonic solutions with prescribed minimal
period for nonautonomous Hamiltonian systems.
J. Diff. Eq. 72, 28--55 (1988)

\bibitem {tar}
Tarantello, G.:
Subharmonic solutions for Hamiltonian
systems via a $\bbbz_{p}$ pseudoindex theory.
Annali di Matematica Pura (to appear)

\bibitem {rab}
Rabinowitz, P.:
On subharmonic solutions of a Hamiltonian system.
Comm. Pure Appl. Math. 33, 609--633 (1980)

\end{thebibliography}
\end{document}
