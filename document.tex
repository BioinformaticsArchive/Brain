\documentclass{bioinfo}
\usepackage{fancyvrb}
\copyrightyear{2005}
\pubyear{2005}

\begin{document}
\firstpage{1}

\title[short Title]{Brain: Biomedical Knowledge Manipulation}
\author[Croset S \textit{et~al}]{Samuel Croset\,$^{1,\footnote{to whom correspondence should be addressed}}$,
John Overington\,$^{1}$ and Dietrich Rebholz-Schuhmann\,$^{1}$}
\address{$^{1}$EMBL European Bioinformatics Institute, Wellcome Trust Genome Campus, Hinxton, Cambridge, CB10 1SD UK}

\history{Received on XXXXX; revised on XXXXX; accepted on XXXXX}

\editor{Associate Editor: XXXXXXX}

\maketitle

\begin{abstract}

\section{Summary:}
Brain is a Java software library facilitating the creation and manipulation of ontologies and knowledge bases represented in the
Web Ontology Language (OWL).

\section{Availability and Implementation:}
The Java source code and the library are freely available at https://github.com/loopasam/Brain
and on the Maven Central repository (GroupId: uk.ac.ebi.brain).
The documentation is available at https://github.com/loopasam/Brain/wiki.

\section{Contact:}
\href{croset@ebi.ac.uk}{croset@ebi.ac.uk}

\section{Supplementary information:}
Supported features and a comparative table are available at the journal's web site. Some supplementary examples are also available
in the online documentation.

\end{abstract}
\section{Motivation}
Knowledge bases, a concept from computer science (see \citealp{Krotzsch2012} for an introduction),
could be a solution to improve the interoperability and the value of the large amount of biomedical information available online.
At the time of writing, a few options are available to handle such knowledge bases: Complex
libraries as the OWL-API \citep{MatthewHorridge2011} or didactic graphical user interfaces such as
Protege \citep{StanfordCenterforBiomedicalInformaticsResearch} and TopBraid Composer (ref). An intermediary framework,
OWLTools \citep{MungallC},
provides some methods to query biomedical ontologies, but the interaction with the library is mostly done via command-lines which
limits the scale of projects that can be build with it.
Brain - the Java software library presented in this manuscript addresses this matter and provides a comprehensive and simplified
interface, dedicated to the programmatic creation and query of biomedical knowledge bases. The library aims at bridging the gap between
graphical user interfaces and the OWL-API and is particularly usefull to develop web applications.
Brain has a particular focus on the EL profile of OWL, as it covers the majority of biomedical
use-cases and unlocks good performances and scalability. 

\section{Scalable knowledge bases}
The Web Ontology Language (OWL) derives from Description Logic and
has been designed to capture the knowledge of a domain of interest in the form of a structured vocabulary \citep{W3COWLWorkingGroup}.
This feature makes it particularly
interesting from the perspective of the life sciences, since a number of ontologies and classification schemes
have been developed from the origin of
the discipline and can now be converted into an OWL representation. Brain focuses on a particular profile of OWL, called EL
which consists of a subset of the constructs available in the original language \citep{Motik2009}. This profile is
designed to be \emph{tractable}, meaning that the axioms available have a polynomial complexity
and are therefore easier to compute than the full version of OWL.
Brain primarily supports the OWL 2 EL profile for its computational properties and suitability for real-life biomedical applications,
where millions of axioms could be potentially extracted from complex repositories such as ChEMBL \citep{Gaulton2012}.
Moreover, the EL profile is expressive enough to cover a good portion of biomedical knowledge: Most of Open Biomedical Ontologies (OBO)
such as the Gene Ontology (GO - \citealp{Ashburner2000}) or
the Chemical Entities of Biological Interest (ChEBI - \citealp{DeMatos2010}) are already included in this
profile, opening doors to large-scale meaningful data integration.
Brain builds on the top of Elk, a fast reasoner dedicated to EL ontologies \citep{YevgenyKazakov2011}. Elk shows good
performances at handling
large datasets and offers the possibility to run some reasoning tasks in parallel; therefore
clusters or multicore architecture can scale the speed of reasoning as more data are added to the knowledge base.
Brain wraps and simplifies the interaction with Elk while still leaving the possibility to fine tune
the configuration for advanced users. Figure 1 shows an example of query using the Elk reasoner.

\section{Library features}
Brain is implemented as a facade leveraging the access to the OWL-API and providing a series of convenience methods for common
use-cases encountered in the biomedical domain. In order to simplify the interaction with the OWL-API, Brain follows
a series of features described below. The full list of currently supported constructs and methods is available in the
 supplementary material and in the online documentation.
\subsection{Unique knowledge base}
An instance of a Brain object hold a reference to only one knowledge base. It is yet possible to import some external ontologies, either
stored locally or via a network but Brain will always merge the added information to the existing knowledge base.
\subsection{Unique short form names}
The names (short forms) of OWL entities handled by a Brain object have to be unique. It is for instance not possible to add
an OWL class such as http://www.example.org/Cell to the ontology if an OWL entity with the short form ``Cell'' already exists.
Despite being in contradiction with some Semantic Web principles, this design prevent ambiguous queries and integration
and hides as much as possible the cumbersome interaction with prefixes and Internationalized Resource Identifiers (IRI).

\begin{figure}[h]
\begingroup
\fontsize{7pt}{8pt}\selectfont
\begin{Verbatim}[frame=single]
//Creation of the Brain instance:
Brain brain = new Brain();
//Add the OWL classes to the knowledge base:
brain.addClass("Nucleus");
brain.addClass("Cell");
//Add the OWL object property:
brain.addObjectProperty("part-of");
//Declare the axiom. Note that expressions in 
//Manchester syntax can be used directly:
brain.subClassOf("Nucleus", "part-of some Cell");
//Integrate the content of external knowledge base:
brain.learn("http://example.org/bar.owl");
//Query the knowledge base for the indirect
//subclasses of an OWL class expression:
List<String> subClasses = 
    brain.getSubClasses("part-of some Cell", false);
//Free the resources used by the reasoner:
brain.sleep();
//Save the ontology:
brain.save("your/path/to/ontology.owl");
\end{Verbatim}
\endgroup
\caption{Implementation example in Java of an axiom using Brain; the axiom expressed in natural language:
 \textit{A nucleus is part of some cells}. Same axiom described in OWL using the Manchester syntax: 
 \textit{Nucleus subClassOf part-of some Cell.}}
\end{figure}

\subsection{Typeless interaction}
The interaction with the library relies on the user-friendly Manchester syntax entered as string \citep{Horridge2006}. 
This choice permits to move away from the creation of Java objects and is particularly suitable in a web server set-up where
requests are likely to be some typeless text. Using strings as input also speeds the production
and flexibility of the code written to move from a relational or flat-file database to an OWL representation for example. Figure 1 provides
an example of axiom implementation using the Manchester syntax.
\subsection{Error-handling}
Because the interaction with Brain is built around strings rather than Java objects, a special care has to be put on
exceptions handling in order to safely maintain the correct execution of the program. Brain throws different types of error tailored 
to the operation performed by the user. This feature is mandatory while developing large applications and helps to maintain the consistency
 of the underlying knowledge base.
\subsection{Knowledge integration}
An interesting feature brought by the Semantic Web and OWL is the possibility to merge information based
on the IRIs of the entities described. The library supports the loading and integration of external knowledge bases as well as 
references to external entities. Data from different sources can therefore be easily connected and reason over by Brain. The integration
of an external knowledge base is shown on Figure 1.
\subsection{Querying}
Brain is oriented towards efficient querying of OWL 2 EL knowledge bases.
This characteristic makes it suitable as query engine on a web server for
answering live queries from users.
Powerful questions can be formulated using the Manchester syntax, abstracting
away complex interaction with the Java objects provided by the OWL-API (illustrated in Figure 1).
An example of question answering over the GO using Brain
is compared against a traditional SQL query in the supplementary material.

\section{Conclusion}
Brain is an open-source Java library designed to build and query biomedical knowledge bases or OWL ontologies.
The library is centered on the EL profile, and designed to be suitable and scalable for biomedical knowledge representation.
The convenience methods provided by Brain should simplify the development of biomedical knowledge bases and allow developers
to increase their productivity while effectively dealing with data integration challenges.

\section*{Acknowledgement}
\paragraph{Funding\textcolon}
This work was supported by funding from the EMBL member states. Samuel Croset is a member of Darwin College, University of Cambridge.

\bibliographystyle{natbib}
\bibliographystyle{achemnat}
\bibliographystyle{plainnat}
\bibliographystyle{abbrv}
\bibliographystyle{bioinformatics}

\bibliographystyle{plain}

 \bibliography{document}

% insert document.bbl here
\end{document}
